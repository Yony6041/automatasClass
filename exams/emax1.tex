% !TEX root = main.tex
\renewcommand{\labelenumi}{\alph{enumi})}

\section*{Examen 1}

\begin{questions}
\question Sean $L_{1} = \{aaa\}^{*},$ $L_{2} = \{wxyz \mid w,x,y,z \in \{a,b\} \}$ y $L_{3} = L_{2}^{*}$.

Describe con palabras (en español) las características de las cadenas en cada uno de estos lenguajes.
Describe las cadenas que contienen los lenguajes $L_{2} \cap L_{3}$ y $L_{1} \cup L_{3}$.

\begin{solution}
    \begin{itemize}
        \item $L_{1} =$ Este lenguaje siempre va a ser de n\'umero impar, contendra
        cadenas concatenadas de 'aaa' multiples veces o no por ende puede la longitud de la cadena es multiple de 3 o bien 0 cuando la cadena es $\epsilon$.

        \item $L_{2} =$ Las cadenas estan formadas por wxyz donde cada uno de esos simbolos contienen ya sea una a o una b por ende la cadea contiene a's, b's o bien a's y b's cabe recalcar que 
        las cadenas pueden tener una longitud max de 4.
        \item $L_{3} =$ Las cadenas en este lenguaje son la concatención de las cadenas formadas por el 
        lenguaje $L_{2}$ o bien $(wxyz)^{n}$ donde n esta en los n\'umeros naturales y w,x,y,z toman el valor de a o b.
        \item $L_{2} \cap L_{3} = $ Todas las cadenas tienen una longitud max de 4 y pueden contener a's, b's o bien ambos o bien nada $\epsilon$.

        \item $L_{1} \cup L_{3} = $ $L_{1}$ esta contenida en $L_{3}$ por ende se pueden formar todas las cadenas que 
        cumplan con las condiciones de $L_{3}$.

    \end{itemize}
    
    
\end{solution} 
\question Sea $L = \{w \in \{a,b\}^{*} \mid \#_{a}(w) \text{ es impar y } \#_{b} (w) \text{ es par} \}$.

Demuestra que $L^{*}$ es el conjunto de cadenas donde el n\'umero de b's es par.

\begin{solution}
    Para solucionar el problema crear\'e dos lenguajes que cumplen
    con las reglas anteriores. \\
    Por ende: \\
    $\Sigma = \{\epsilon, a,b\}$

    $L_{1} = \{w \in \Sigma^{*}\mid \#_{a}(w)$ es impar $\}$
    $L_{2} = \{w \in \Sigma^{*}\mid \#_{b}(w)$ es par \} \\

    Por ende podriamos reescribir L como: \\
    $L = \{w \in \{a,b\}^{*} \mid w \in (L_{1} \cap L_{2})\}$
    \\

    y a $L^{*}$ como $L^{*} = \cup^{\infty}_{i=0} L^{i} = L^{0} \cup L^{1} \cup L^{2} \cup L^{3} ... L^{\infty}$ 
    o bien \\ $L^{*} = \cup^{\infty}_{i=0} L^{i} = L^{0} \cup L \cup LL \cup LLL ... L^{\infty}$

    En el caso base para el lenguaje L, w = $\epsilon$ en este caso el numero de b's es par ya que 0 es par, si procedemos a armar una cadena 
    no vacia por definicion $L = \{w \in \{a,b\}^{*} \mid w \in (L_{1} \cap L_{2})\}$ ya que w debe cumplir con las condiciones
    para $L_{2}$ el numero de b's ser\'a par tambien por ende no se pueden crear cadenas con L donde w tenga un numero de b's impar \\

    Ya que todo numero par sumado con otro par sigue siendo par y ya que la cantidad de a's no afecta a las propiedades de $L_{2}$
    gracias a la intersecci\'on $\{w \in (L_{1} \cap L_{2})\}$, podemos asegurar que cualquier concatenacion de una cadena w consigo misma $w^n$ seguir\'a teniendo 
    un numero par de b's. \\
    $\therefore L^{*}$  siempre tendra un numero par de b's para cualquier concatenacion de L's.

    
\end{solution} 

\question Sean $\Sigma = \{a,b\}$  y L el lenguaje que contiene a todas las cadenas en cada
uno de estos lenguajes tales que no terminan en b y no tienen a bb como subcadena.

Define de forma {\bf no} recursiva a un lenguaje S de modo que $L = S^{*}$ es decir usando lenguajes
y operaciones entre ellos.

\begin{solution}
    \\
    \begin{enumerate}
        \item Decimos que v es una subcadena de u si existen cadenas $x,y \in \Sigma^{*}$ tales que u = xvy.
    \end{enumerate}
\end{solution}

\question La siguiente notaci\'on $|L_{1}L_{2}| = |L_{1}||L_{2}|$ denota que el {\it número de cadenas en la concatención}
$L_{1}L_{1}$ {\it es el mismo que el producto de dos n\'umeros $|L_{1}|$} y $|L_{2}|$.

Si esta afirmaci\'on siempre es verdad para cualesquiera dos lenguajes, da argumentos
formales para probarlo y si no, muestra dos lenguajes $|L_{1}|$ y $|L_{2}|$ tales que $|L_{1}L_{2}| \neq |L_{1}||L_{2}|$ (para el
contraejemplo, puedes considerar que los lenguajes pertenecen a $\Sigma^{*}$ donde $\Sigma = \{a,b\}$).

\begin{solution}
    Antes que nada definamos la concatenacion de lenguajes.
    \begin{equation}
        A \cdot B = AB = \{ wx \mid w \in A \wedge x \in B\}
    \end{equation}
    Para el caso base imaginemos que $L_{2} = \{\epsilon\}$ Por ende $L_{1}L_{2} = L_{1}\{\epsilon\}$

    $|L_{1}\{\epsilon\}| = |L_{1}|$ 

    Si vemos la longitud de $\{\epsilon\} = 1$ Por ende \\
    $|L_{1}\{\epsilon\}| = |L_{1}| \cdot 1 = L_{1}$

    Ya habiendo cubierto el caso base ahora\dots \\

    $(\{w\}L_{1})L_{2} = |{w}(L_{1}L_{2})| = 1 \cdot |L_{1}L_{2}|$

    $|L_{1}||L_{2}| = |(\{w\}L_{1})| \cdot |L_{2}| = 1 \cdot |L_{2}| \cdot |L_{2}|$

    $\therefore$ La hipotesis de inducci\'on se cumple correctamente.
    
\end{solution}
\end{questions}


