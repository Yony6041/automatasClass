% !TEX root = main.tex
\renewcommand{\labelenumi}{\alph{enumi})}

\section*{Examen 1}

\begin{questions}
\question Sean $L_{1} = \{aaa\}^{*},$ $L_{2} = \{wxyz \mid w,x,y,z \in \{a,b\} \}$ y $L_{3} = L_{2}^{*}$.

Describe con palabras (en español) las características de las cadenas en cada uno de estos lenguajes.
Describe las cadenas que contienen los lenguajes $L_{2} \cap L_{3}$ y $L_{1} \cup L_{3}$.

\begin{solution}
    
    
    
\end{solution} 
\question Sea $L = \{w \in \{a,b\}^{*} \mid \#_{a}(w) \text{ es impar y } \#_{b} (w) \text{ es par} \}$.

Demuestra que $L^{*}$ es el conjunto de cadenas donde el n\'umero de b's es par.

\begin{solution}
    Para solucionar el problema crear\'e dos lenguajes que cumplen
    con las reglas anteriores. \\
    Por ende: \\
    $\Sigma = \{\epsilon, a,b\}$

    $L_{1} = \{w \in \Sigma^{*}\mid \#_{a}(w)$ es impar \} \\
    $L_{2} = \{w \in \Sigma^{*}\mid \#_{b}(w)$ es par \} \\

    Por ende podriamos reescribir L como: \\
    $L = \{w \in \{a,b\}^{*} \mid w \in (L_{1} \cup L_{2})\}$

    Si 

    
\end{solution} 

\question Sean $\Sigma = \{a,b\}$  y L el lenguaje que contiene a todas las cadenas en cada
uno de estos lenguajes tales que no terminan en b y no tienen a bb como subcadena.

Define de forma {\bf no} recursiva a un lenguaje S de modo que $L = S^{*}$ es decir usando lenguajes
y operaciones entre ellos.

\question La siguiente notaci\'on $|L_{1}L_{2}| = |L_{1}||L_{2}|$ denota que el {\it número de cadenas en la concatención}
$L_{1}L_{1}$ {\it es el mismo que el producto de dos n\'umeros $|L_{1}|$} y $|L_{2}|$.

Si esta afirmaci\'on siempre es verdad para cualesquiera dos lenguajes, da argumentos
formales para probarlo y si no, muestra dos lenguajes $|L_{1}|$ y $|L_{2}|$ tales que $|L_{1}L_{2}| \neq |L_{1}||L_{2}|$ (para el
contraejemplo, puedes considerar que los lenguajes pertenecen a $\Sigma^{*}$ donde $\Sigma = \{a,b\}$).
\end{questions}
